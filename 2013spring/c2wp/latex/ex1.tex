%@+leo-ver=5-thin
%@+node:amd_yen.20130503185304.1804: * @file c2wp/latex/ex1.tex
%@@language latex
%@+<< document class >>
%@+middle:amd_yen.20130503185304.1817: ** 相關設定
%@+node:amd_yen.20130503185304.1824: *3* << document class >>
% The first line \documentclass{article} tells LATEX that what we want to produce is an article.
% 假如要使用雙欄位
\documentclass[twocolumn]{article}
%\documentclass{article}
%@-<< document class >>
%@+<< CJK >>
%@+middle:amd_yen.20130503185304.1817: ** 相關設定
%@+node:amd_yen.20130503185304.1823: *3* << CJK >>
\usepackage{xeCJK}    %使用中文環境
\usepackage[T1]{fontspec}    %設定字體用
\usepackage{graphicx}
\usepackage{fancyvrb} % for frame on Verbatim
% 新細明體
% 文鼎PL新宋
\setCJKmainfont{新細明體}



%@-<< CJK >>
%@+<< author title >>
%@+middle:amd_yen.20130503185304.1817: ** 相關設定
%@+node:amd_yen.20130503185304.1822: *3* << author title >>
% 加入標題與作者資料, 然後在文章中的 maketitle 中才會顯示出來
\title{就是要學 \LaTeX}
\author{小作者 \and 大作者\footnote{小作者為大老師,
and 大作者為小學生.}}
\date{\today}
%@-<< author title >>
%@+<< verbatim >>
%@+middle:amd_yen.20130503185304.1817: ** 相關設定
%@+node:amd_yen.20130503185304.1821: *3* << verbatim >>
% 為了讓 verbatim 註解中能夠加入中文, 採用 roman family 字體
\makeatletter
\def\verbatim@font{\rmfamily\small}
\makeatother
%@-<< verbatim >>
%@+<< begin document >>
%@+middle:amd_yen.20130503185304.1817: ** 相關設定
%@+node:amd_yen.20130503185304.1820: *3* << begin document >>
\begin{document}
%@-<< begin document >>
%@+<< make title >>
%@+middle:amd_yen.20130503185304.1817: ** 相關設定
%@+node:amd_yen.20130503185304.1819: *3* << make title >>
\maketitle
%@-<< make title >>
%@+others
%@+node:amd_yen.20130503185304.1817: ** 相關設定
%@+node:amd_yen.20130503185304.1813: ** 內文
This is my \emph{first} document prepared in \LaTeX.
This is my \emph{first} document prepared in \LaTeX. I typed it on \today.

\noindent We have seen that to typeset something in \LaTeX, we type in the
text to be typeset together with some \LaTeX\ commands.
Words must be separated by spaces (does not matter how many)
and lines maybe broken arbitrarily.
The end of a paragraph is specified by a \emph{blank line}
in the input. In other words, whenever you want to start a new
paragraph, just leave a blank line and proceed.

Carrots are good for your eyes, since they contain Vitamin A\@. Have
you ever seen a rabbit wearing glasses?

Carrots are good for your eyes, since they contain Vitamin A. Have
you ever seen a rabbit wearing glasses?

The numbers 1, 2, 3, etc.\ are called natural numbers. According to
Kronecker, they were made by God;all else being the works of Man.
I think \LaTeX is fun.
I think \LaTeX\ is fun.
Note the difference in right and left quotes in `single quotes'
and ``double quotes''.

Note the difference in right and left quotes in \lq single
quotes\rq\ and \lq\lq double quotes\rq\rq.
X-rays are discussed in pages 221--225 of Volume 3---the volume on
electromagnetic waves.
\`{E}l est\`{a} aqu\`{\i}
Maybe I have now learnt about 1\% of \LaTeX.
以下為特殊符號的輸入方法:以下為特殊符號的輸入方法
以下為特殊符號的輸入方法以下為特殊符號的輸入方法
以下為特殊符號的輸入方法以下為特殊符號的輸入方法
以下為特殊符號的輸入方法以下為特殊符號的輸入方法
以下為特殊符號的輸入方法
\textasciitilde  \&
 \# \_
\$ \textbackslash
 \%  \{
\textasciicircum 
 \}
 
We can also give an optional argument to \\ to increase the vertical distance between the
lines. For example,
This is the first line.\\[10pt]
This is the second line

Now there is an extra 10 points of space between the lines (1point is about 1/72nd of an
inch).

\begin{center}
The \TeX nical Institute\\[.75cm]
Certificate 證書
\end{center}
\noindent This is to certify that Mr. N. O. Vice has undergone a
course at this institute and is qualified to be a \TeX nician.\\[.75cm]
\noindent 沒有錯, 這裡要證明他真的要下台了.

\begin{flushright}
The Director\\
The \TeX nical Institute
\end{flushright}

\rule{80mm}{.2pt}
% 要讓 fbox 可以旋轉, 必須要導入 graphicx package
\rotatebox{45}{\fbox{\LARGE{\LaTeX}}}\\[2cm]
\rotatebox{45}{\fbox{\LARGE{真的可以轉}}}
\rotatebox{225}{\fbox{\LARGE{真的可以轉}}}\\[2cm]

文字之間的數學

The equation $x + 2y = 3$ is a linear function,
its graph is a line.
The equation $y = (x-1)^2 + 2$ is a quadratic function,
its graph is a parabola.

The price of the 大爛書 textbook is \$500 NT dollars.

還有以下的數學符號

If the following limit exists:
$$
\lim_{\Delta x\to 0} \frac{f(x_0+\Delta x) - f(x_0)}{\Delta x}
$$
The limit is called {\it the derivative of $f(x)$ at $x_0$},
denoted by $f'(x)$.

準備要列舉

\begin{itemize}
\item Calculus
\item Linear Algebra
\item Basic Computer Concepts
\end{itemize}

還是要列舉

\begin{enumerate}
\item Calculus
\item Linear Algebra
\item Basic Computer Concepts
\end{enumerate}

也是來列舉

\begin{description}
\item [Calculus] Integration and Differentiation
\item [Linear Algebra] Vector space and Basis
\item [Basic Computer Concepts] Programming Language
\end{description}

自己定指令

\newcommand{\nfu}{國立虎尾小科技大學}

我們的學校就是 \nfu

% 要讓 Verbatim 發揮作用需要 \usepackage{fancyvrb} % for frame on Verbatim
% frame=single 會以四邊框起, 若以 frame=lines 會以最長程式為主
\begin{Verbatim}[frame=single, numbers=left]
# 請注意程式內縮
for i in range(4):
    print('Hello!')
    print("應該沒有問題")
\end{Verbatim}
%@+node:amd_yen.20130503185304.1814: ** 測試
這是第一行\\
這是第二行, 所以兩個反斜線就是強制跳行\\
\framebox [5cm][r]{這是裡面的字串}\\
\framebox [5cm][c]{這是裡面的字串}\\
\framebox [5cm][l]{這是裡面的字串}\\
{\centering
這些內容可以置中這些內容可以置中% 請注意下面一行空白非常重要, 否則無法讓 centering 執行

}

X  X\\
X\enspace  X\\
%@+node:amd_yen.20130503185304.1815: ** 以下為來自外部的檔案
%@+node:amd_yen.20130503185304.1816: ** @auto latex/outside.tex
這是來自外部的檔案資料

也是來列舉用的

\begin{itemize}
\item 微積分
\item 線性代數
\item 電腦小概念
\end{itemize}
%@+node:amd_yen.20130503185304.1817: ** 相關設定
%@-others
%@+<< end document >>
%@+middle:amd_yen.20130503185304.1817: ** 相關設定
%@+node:amd_yen.20130503185304.1818: *3* << end document >>
\end{document}
%@-<< end document >>
%@-leo

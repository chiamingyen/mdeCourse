%%%%%%%%%%%%%%%%%%%%%%%%%%%%%%%%%%%%%%%%%%%%%%%%%%%%%%%%%%%%%%%%%
% Contents: Who contributed to this Document
% $Id: overview.tex,v 1.2 2003/03/19 20:57:46 oetiker Exp $
%%%%%%%%%%%%%%%%%%%%%%%%%%%%%%%%%%%%%%%%%%%%%%%%%%%%%%%%%%%%%%%%%
% 中文~4.20~翻譯:zpxing@bbs.ctex  email: zpxing at gmail dot com
%%%%%%%%%%%%%%%%%%%%%%%%%%%%%%%%%%%%%%%%%%%%%%%%%%%%%%%%%%%%%%%%%

% Because this introduction is the reader's first impression, I have
% edited very heavily to try to clarify and economize the language.
% I hope you do not mind! I always try to ask "is this word needed?"
% in my own writing but I don't want to impose my style on you...
% but here I think it may be more important than the rest of the book.
% --baron

%\chapter{Preface}
\chapter{前言}

%\LaTeX{} \cite{manual} is a typesetting system that is very suitable
%for producing scientific and mathematical documents of high
%typographical quality. It is also suitable for producing all sorts
%of other documents, from simple letters to complete books. \LaTeX{}
%uses \TeX{} \cite{texbook} as its formatting engine.
\LaTeX{}~\cite{manual}~是一種排版系統,它非常適合用來生成高印刷質量
的科技和數學類文檔。這個系統同樣適用於生成從簡單信件到完整書籍
的所有其他種類的文檔。\LaTeX~使用~\TeX{}\cite{texbook}~作為它的
格式化引擎。%

%This short introduction describes \LaTeXe{} and should be sufficient
%for most applications of \LaTeX. Refer to~\cite{manual,companion}
%for a complete description of the \LaTeX{} system.
這份簡短的介紹描述了~\LaTeXe{}~的使用,對~\LaTeX{}~的大多數應用來說
應該是足夠了。參考文獻~\cite{manual,companion}~對~\LaTeX{}~系統
提供了完整的描述。%

\bigskip
%\noindent This introduction is split into 6 chapters:
\noindent 這份介紹共有六章:%
%\begin{description}
%\item[Chapter 1] tells you about the basic structure of \LaTeXe{}
%  documents. You will also learn a bit about the history of \LaTeX{}.
%  After reading this chapter, you should have a rough understanding how
%  \LaTeX{} works.
%\item[Chapter 2] goes into the details of typesetting your
%  documents. It explains most of the essential \LaTeX{} commands and
%  environments. After reading this chapter, you will be able to write
%  your first documents.
%\item[Chapter 3] explains how to typeset formulae with \LaTeX. Many
%  examples demonstrate how to use one of \LaTeX{}'s
%  main strengths. At the end of the chapter are tables listing
%  all mathematical symbols available in \LaTeX{}.
%\item[Chapter 4] explains indexes,  bibliography generation and
%  inclusion of EPS graphics. It introduces creation of PDF documents with pdf\LaTeX{}
%  and presents some handy extension packages.
%\item[Chapter 5] shows how to use \LaTeX{} for creating graphics. Instead
%  of drawing a picture with some graphics program, saving it to a file and
%  then including it into \LaTeX{} you describe the picture and have \LaTeX{}
%  draw it for you.
%\item[Chapter 6] contains some potentially dangerous information about
%  how to alter the
%  standard document layout produced by \LaTeX{}. It will tell you how  to
%  change things such that the beautiful output of \LaTeX{}
%  turns ugly or stunning, depending on your abilities.
%\end{description}
\begin{description}%
\item[第一章] 告訴你關於~\LaTeXe{}~文檔的基本結構。你也會從中瞭解一點~\LaTeX{}~的歷史。
              閱讀這一章後,你應該對~\LaTeX{}~如何工作有一個大致的理解。
\item[第二章] 探究文檔排版的細節。它解釋了大部分必要的~\LaTeX{}~命令和環境。在閱
              讀完這一章之後,你就能夠編寫你的第一份文檔了。%
\item[第三章] 解釋了如何使用~\LaTeX{}~排版公式。同時,大量的例子會有助於你理解
              ~\LaTeX{}~是如何的強大。在這個章節的結尾,你會找到列出~\LaTeX{}~
              中所有可用數學符號的表格。%
\item[第四章] 解釋了索引和參考文件的生成、EPS~圖形的插入。它介紹了如何使用~pdf\LaTeX{}~生成~pdf~文檔和一些其他有用的擴展宏包。%
\item[第五章] 演示如何使用~\LaTeX{}~創建圖形。不必使用圖形軟件畫圖、存檔並插入~\LaTeX{}~文檔,你可以直接描述
              圖形,然後~\LaTeX{}~會替你畫好它。
\item[第六章]
包含一些潛在的危險信息,內容是關於如何改變~\LaTeX{}~所產生文檔的標準佈局。它會告訴你如何把~\LaTeX{}~的輸出
變得更糟糕,或者更上一層樓,當然這取決於你的能力。
\end{description}%

\bigskip
%\noindent It is important to read the chapters in order---the book is
%not that big, after all. Be sure to carefully read the examples,
%because a lot of the information is in the
%examples placed throughout the book.
\noindent 按照順序閱讀這些章節是很重要的
\pozhehao 這本書畢竟不長。一定要認真閱讀例子,因為在貫穿全篇的各種例子裡包含了很多的信息。%

\bigskip
%\noindent \LaTeX{} is available for most computers, from the PC and Mac to large
%UNIX and VMS systems. On many university computer clusters you will
%find that a \LaTeX{} installation is available, ready to use.
%Information on how to access
%the local \LaTeX{} installation should be provided in the \guide. If
%you have problems getting started, ask the person who gave you this
%booklet. The scope of this document is \emph{not} to tell you how to
%install and set up a \LaTeX{} system, but to teach you how to write
%your documents so that they can be processed by~\LaTeX{}.
\noindent
\LaTeX{}~適用於從~PC~和~Mac~到大型的~UNIX~和~VMS~系統上。許多大學的計算機集群上安
裝了~\LaTeX{},隨時可以使用。\guide~裡應該會介紹如何使用本地安裝的~\LaTeX{}。
如果有問題,就去問給你這本小冊子的人。這份文檔\emph{不}會告訴你如何安裝一個~\LaTeX{}~系統,
而是教會你編寫~\LaTeX{}~能夠處理的文檔。

\bigskip
%\noindent If you need to get hold of any \LaTeX{} related material,
%have a look at one of the Comprehensive \TeX{} Archive Network
%(\texttt{CTAN}) sites. The homepage is at
%\texttt{http://www.ctan.org}. All packages can also be retrieved from
%the ftp archive \texttt{ftp://www.ctan.org} and its mirror
%sites all over the world.
\noindent 如果你想取得~\LaTeX{}~的相關材料,請訪問「Comprehensive
\TeX{} Archive Network」
~(\texttt{CTAN})~站點,主頁是~\texttt{http://www.ctan.org}。所有的宏包
也可以從~ftp~歸檔站點~\texttt{ftp://www.ctan.org}~和遍佈全球的各個鏡像站點中獲得。
所有的宏包都可以在~\texttt{ftp://ctan.tug.org}~以及它遍佈全球的鏡像取得。

%You will find other references to CTAN throughout the book, especially
%pointers to software and documents you might want to download. Instead
%of writing down complete urls, I just wrote \texttt{CTAN:} followed by
%whatever location within the CTAN tree you should go to.
在本書中你會找到其他引用~\texttt{CTAN}~的地方,尤其是,給出你可能需要下載的軟件和文檔的
指示。這裡沒有寫出完整的~url,而僅僅是其在~\texttt{CTAN:}~之後的樹狀結構中的位置。%


%If you want to run \LaTeX{} on your own computer, take a look at what
%is available from \CTAN|systems|.
請先看看~\CTAN|systems|~中有些什麼,如果你想在自己的計算機上運行~\LaTeX{}。%

\vspace{\stretch{1}}
%\noindent If you have ideas for something to be
%added, removed or altered in this document, please let me know. I am
%especially interested in feedback from \LaTeX{} novices about which
%bits of this intro are easy to understand and which could be explained
%better.
\noindent 如果你有意在這份文檔中增加、刪除或者改變一些內容,請通知我。我對~\LaTeX{}~
初學者的反饋特別感興趣,尤其是關於這份介紹哪些部分很容易理解,哪些部分可能需要更好地解釋。%
%

\bigskip
\begin{verse}
\contrib{Tobias Oetiker}{oetiker@ee.ethz.ch}%
\noindent{Department of Information Technology and\\ Electrical
Engineering,\\
Swiss Federal Institute of Technology}
\end{verse}
\vspace{\stretch{1}}

%\noindent The current version of this document is available on\\
%\CTAN|info/lshort|

\noindent 這份文檔的最新版本在\\
\CTAN|info/lshort|\\%

\smallskip
\noindent 關於這份文檔的最新中文翻譯,請諮詢\\
\href{http://bbs.ctex.org}{\texttt{http://bbs.ctex.org}}\\

\endinput



%

% Local Variables:
% TeX-master: "lshort2e"
% mode: latex
% mode: flyspell
% End:


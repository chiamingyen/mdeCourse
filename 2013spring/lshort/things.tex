%%%%%%%%%%%%%%%%%%%%%%%%%%%%%%%%%%%%%%%%%%%%%%%%%%%%%%%%%%%%%%%%%
% Contents: Things you need to know
% $Id: things.tex,v 1.2 2003/03/19 20:57:47 oetiker Exp $
%%%%%%%%%%%%%%%%%%%%%%%%%%%%%%%%%%%%%%%%%%%%%%%%%%%%%%%%%%%%%%%%%
%中文~4.20~翻譯:Frogge@bbs.ctex
%%%%%%%%%%%%%%%%%%%%%%%%%%%%%%%%%%%%%%%%%%%%%%%%%%%%%%%%%%%%%%%%%
%\chapter{Things You Need to Know}
\chapter{基礎知識}
%\begin{intro}
%The first part of this chapter presents a short overview of the
%philosophy and history of \LaTeXe. The second part focuses on the
%basic structures of a \LaTeX{} document. After reading this chapter,
%you should have a rough knowledge of how \LaTeX{} works, which you
%will need to understand the rest of this book.
%\end{intro}
\begin{intro}
本章的第一部分給出了 \LaTeXe{} 原理及歷史的簡短介紹。第二部分集中講
解 \LaTeX{} 文檔的基本結構。讀完本章之後,你應該大致瞭解 \LaTeX{} 的工作原理,這對你理解
本書的其餘部分來說是必須的。
\end{intro}

%\section{The Name of the Game}
\section{遊戲的名目}
%\subsection{\TeX}
\subsection{\TeX}

%\TeX{} is a computer program created by \index{Knuth, Donald
%E.}Donald E. Knuth \cite{texbook}. It is aimed at typesetting text
%and mathematical formulae. Knuth started writing the \TeX{}
%typesetting engine in 1977 to explore the potential of the digital
%printing equipment that was beginning to infiltrate the publishing
%industry at that time, especially in the hope that he could reverse
%the trend of deteriorating typographical quality that he saw
%affecting his own books and articles. \TeX{} as we use it today was
%released in 1982, with some slight enhancements added in 1989 to
%better support 8-bit characters and multiple languages. \TeX{} is
%renowned for being extremely stable, for running on many different
%kinds of computers, and for being virtually bug free. The version
%number of \TeX{} is converging to $\pi$ and is now at $3.141592$.

\TeX{} 是 \index{Knuth, Donald E.}Donald E.
Knuth 編寫的一個以排版文章及數學公式為目標的計算機程序 \cite{texbook}。1977 年,在意識到惡劣的排版質量正在影響自己的著
作及文章後,Knuth 開始編寫 \TeX{} 排版系統引擎,探索當時開始進入出版工業的數字印刷設備的潛力,尤為希望能扭轉排版質量下滑
的這一趨勢。我們現在使用的 \TeX{} 系統發佈於 1982 年,在 1989 年又稍做改進,增加了對 8 字節字符及多語言的支
持。\TeX{} 以其卓越的穩定性、可在不同類型的電腦上運行以及幾乎沒有缺
陷而著稱。\TeX{} 的版本號不斷趨近於 $\pi$,現在為 3.141592。

%\TeX{} is pronounced ``Tech,'' with a ``ch'' as in the German word
%``Ach''\footnote{In german there are actually two pronounciations
%for ``ch'' and one might assume that the soft ``ch'' sound from
%``Pech'' would be a more appropriate. Asked about this, Knuth wrote
%in the German Wikipedia: \emph{I do not get angry when people
%pronounce \TeX{} in their favorite way \ldots{} and in Germany many
%use a soft ch because the X follows the vowel e, not the harder ch
%that follows the vowel a. In Russia, `tex' is a very common word,
%pronounced `tyekh'. But I believe the most proper pronunciation is
%heard in Greece, where you have the harsher ch of ach and Loch.}} or
%in the Scottish ``Loch.'' The ``ch'' originates from the Greek
%alphabet where X is the letter ``ch'' or ``chi''. \TeX{} is also the
%first syllable of the Greek word texnologia (technology). In an
%ASCII environment, \TeX{} becomes \texttt{TeX}.

\TeX{} 發音為 ``Tech'',其中 ``ch'' 和德語 ``Ach''\footnote{在德語中,``ch'' 有兩種發音,有的人可能認為 ``Pech'' 中
較軟的 ``ch'' 更加合適。被問及這個問題時,Knuth 在德文 Wikipedia 中寫道:{
當人們以他們喜歡的方式來拼讀 \TeX{} 時,我並不感到生氣……在德國,更多的人喜歡較軟的 ch,因為 X 跟
在元音 e 的後面。在俄語中,`tex' 是一個非常普遍的單詞,讀作 `tyekh'。但我相信最合適的發音來自希臘
語,其中 ach 和 Loch 中 ch 的發音稍尖。}} 及蘇格蘭語 ``Loch'' 中的 ``ch'' 類似。``ch'' 源自希臘字母,希臘文中,X 是字
母 ``ch'' 或 ``chi''。 \TeX{} 同時也是希臘單詞 texnologia (technology) 的第一個音節。在 \texttt{ASCII} 文本環
境中,\TeX{} 寫作 \texttt{TeX}。


\subsection{\LaTeX}
%\LaTeX{} is a macro package that enables authors to typeset and
%print their work at the highest typographical quality, using a
%predefined, professional layout. \LaTeX{} was originally written by
%\index{Lamport, Leslie}Leslie Lamport \cite{manual}. It uses the
%\TeX{} formatter as its typesetting engine. These days \LaTeX{} is
%maintained by \index{Mittelbach, Frank}Frank Mittelbach.

\LaTeX{} 是一個宏集,它使用一個預先定義好的專業版面,可以使作者們高質量的排版和列印他們的作品。\LaTeX{} 最初
由 \index{Lamport, Leslie}Leslie
Lamport 編寫 \cite{manual},它使用 \TeX{} 程序作為排版引擎。現
在 \LaTeX{} 由 \index{Mittelbach, Frank}Frank Mittelbach 負責維護。

%In 1994 the \LaTeX{} package was updated by the \index{LaTeX3@\LaTeX
%  3}\LaTeX 3 team, led by \index{Mittelbach, Frank}Frank Mittelbach,
%to include some long-requested improvements, and to re\-unify all the
%patched versions which had cropped up since the release of
%\index{LaTeX 2.09@\LaTeX{} 2.09}\LaTeX{} 2.09 some years earlier. To
%distinguish the new version from the old, it is called \index{LaTeX
%2e@\LaTeXe}\LaTeXe. This documentation deals with \LaTeXe. These days you
%might be hard pressed to find the venerable \LaTeX{} 2.09 installed
%anywhere.

%\LaTeX{} is pronounced ``Lay-tech'' or ``Lah-tech.'' If you refer to
%\LaTeX{} in an \texttt{ASCII} environment, you type \texttt{LaTeX}.
%\LaTeXe{} is pronounced ``Lay-tech two e'' and typed
%\texttt{LaTeX2e}.

\LaTeX{} 的發音為 ``Lay-tech'' 或 ``Lah-tech''。
如果在 \texttt{ASCII} 環境中引用 \LaTeX{},你可以輸入 \texttt{LaTeX}。
\LaTeXe{} 的發音為 ``Lay-tech two e'',在 \texttt{ASCII} 環境中寫作 \texttt{LaTeX2e}。


%Figure \ref{components} above % on page \pageref{components}
%shows how \TeX{} and \LaTeXe{} work together. This figure is taken from
%\texttt{wots.tex} by Kees van der Laan.

%\begin{figure}[btp]
%\begin{lined}{0.8\textwidth}
%\begin{center}
%\input{kees.fig}
%\end{center}
%\end{lined}
%\caption{Components of a \TeX{} System.} \label{components}
%\end{figure}

%\section{Basics}
\section{基礎}

%\subsection{Author, Book Designer, and Typesetter}
\subsection{作者、圖書設計者和排版者}
%To publish something, authors give their typed manuscript to a
%publishing company. One of their book designers then
%decides the layout of the document (column width, fonts, space before
%and after headings, \ldots). The book designer writes his instructions
%into the manuscript and then gives it to a typesetter, who typesets the
%book according to these instructions.

出版的第一步就是作者把打好字的手稿交給出版公司,然後由圖書設計者來決定整個文檔的佈局(欄寬、字體、標題前後的間距、……)。圖書
設計者會把他的排版說明寫進作者的手稿裡,再交給排版者,由排版者根
據這些說明來排版全書。

%A human book designer tries to find out what the author had in mind
%while writing the manuscript. He decides on chapter headings,
%citations, examples, formulae, etc.\ based on his professional
%knowledge and from the contents of the manuscript.

一個圖書設計者要試圖理解作者寫作時的意圖。他要根據手稿的內容
和他自己的職業知識來決定章節標題、文獻引用、例子及公式等等。

%In a \LaTeX{} environment, \LaTeX{} takes the role of the book
%designer and uses \TeX{} as its typesetter. But \LaTeX{} is ``only'' a
%program and therefore needs more guidance. The author has to provide
%additional information to describe the logical structure of his
%work. This information is written into the text as ``\LaTeX{}
%commands.''

在一個 \LaTeX{} 環境中,\LaTeX{} 充當了圖書設計者的角色,而 \TeX{} 則是其排版者。但是 \LaTeX 「僅僅」是
一個程序,因此它需要很多的指導。作者必須提供額外的信息,來描述其著作的邏輯結構。這些信息是以 「\LaTeX{} 命令」 的形
式寫入文檔中的。

%This is quite different from the \wi{WYSIWYG}\footnote{What you see is
%  what you get.} approach that most modern word processors, such as
%\emph{MS Word} or \emph{Corel WordPerfect}, take. With these
%applications, authors specify the document layout interactively while
%typing text into the computer. They can see on the
%screen how the final work will look when it is printed.
\hyphenation{WordPerfect}
這和大多數現代文字處理工具,如 \emph{MS Word} 及 \emph{Corel
WordPerfect} 所採用的所見即所得 (\wi{WYSIWYG}\footnote{What you see
is what you
get.}) 的方式有很大區別。使用這些工具時,作者在向計算機中輸入文檔的同時,通過互
動的方式確定文章的佈局。作者可以從螢幕上看到作品的最終列印效果。

%When using \LaTeX{} it is not normally possible to see the final output
%while typing the text, but the final output can be previewed on the
%screen after processing the file with \LaTeX. Then corrections can be
%made before actually sending the document to the printer.

而使用 \LaTeX 時,一般是不能在輸入文檔的同時看到最終的輸出效果的,但是使用 \LaTeX 處理文檔之後,便可以在螢幕上預覽
最終的輸出效果。因此在真正列印文檔之前還是可以做出改正的。

%\subsection{Layout Design}
\subsection{版面設計}

%Typographical design is a craft. Unskilled authors often commit
%serious formatting errors by assuming that book design is mostly a
%question of aesthetics---``If a document looks good artistically,
%it is well designed.'' But as a document has to be read and not hung
%up in a picture gallery, the readability and understandability is
%much more important than the beautiful look of it.
%Examples:
%\begin{itemize}
%\item The font size and the numbering of headings have to be chosen to make
%  the structure of chapters and sections clear to the reader.
%\item The line length has to be short enough not to strain
%  the eyes of the reader, while long enough to fill the page
%  beautifully.
%\end{itemize}

排版設計是一門工藝。不熟練的作者認為書籍設計僅僅是個美學問題,因而經常會犯嚴重的格式錯誤 \pozhehao 「如果一份文檔從藝術的
角度看起來不錯,那麼它的設計就是成功的」。不過作為一份用來閱讀而不是掛在畫廊裡的文檔,可讀性和可理解性遠比漂亮的
外觀重要。例如:
\begin{itemize}
\item 必須選定字號和標題的序號,使讀者能清楚的理解章節的結構。
\item
每一行既要足夠短以避免讀者眼睛疲勞,又要足夠長以維持頁面的美觀。
\end{itemize}

%With \wi{WYSIWYG} systems, authors often generate aesthetically
%pleasing documents with very little or inconsistent structure.
%\LaTeX{} prevents such formatting errors by forcing the author to
%declare the \emph{logical} structure of his document. \LaTeX{} then
%chooses the most suitable layout.

在使用所見即所得系統 (\wi{WYSIWYG}) 時,作者經常會寫出一些看上去漂亮,但結構欠清晰或不連貫的文章來。\LaTeX{} 通過強制
作者聲明文檔的\textbf{邏輯}結構,來避免這些排版格式錯誤。然後,\LaTeX{} 再根據文檔的結構選擇最合適的版面格式。

%\subsection{Advantages and Disadvantages}
\subsection{優勢和不足}

%When people from the \wi{WYSIWYG} world meet people who use \LaTeX{},
%they often discuss ``the \wi{advantages of \LaTeX{}} over a normal
%word processor'' or the opposite.  The best thing you can do when such
%a discussion starts is to keep a low profile, since such discussions
%often get out of hand. But sometimes you cannot escape \ldots

使用所見即所得 (\wi{WYSIWYG}) 的人和使用 \LaTeX{} 的人遇到一起時,他們經常討論的話題
就是「相比一般文字處理軟件,\LaTeX{} 的優勢 (\wi{advantages of
\LaTeX{}})」或者不足。當
這樣的討論開始時,你最好保持低調,因為討論往往會失控。但有時你也不能逃避……

%\medskip\noindent So here is some ammunition. The main advantages
%of \LaTeX{} over normal word processors are the following:

下面便是一些武器。\LaTeX{} 優於一般文字處理軟件之處可歸納如下:

%\begin{itemize}
%
%\item Professionally crafted layouts are available, which make a
%  document really look as if ``printed.''
%\item The typesetting of mathematical formulae is supported in a
%  convenient way.
%\item Users only need to learn a few easy-to-understand commands
%  that specify the logical structure of a document. They almost never
%  need to tinker with the actual layout of the document.
%\item Even complex structures such as footnotes, references, table of
%  contents, and bibliographies can be generated easily.
%\item Free add-on packages exist for many typographical tasks not directly supported by basic
%  \LaTeX. For example, packages are
%  available to include \PSi{} graphics or to typeset
%  bibliographies conforming to exact standards. Many of these add-on
%  packages are described in \companion.
%\item \LaTeX{} encourages authors to write well-structured texts,
%  because this is how \LaTeX{} works---by specifying structure.
%\item \TeX, the formatting engine of \LaTeXe, is highly portable and free.
%  Therefore the system runs on almost any hardware platform
%  available.
%
%%
%% Add examples ...
%%
%\end{itemize}

\begin{itemize}

\item 提供專業的版面設計,可以使一份文檔看起來就像「印刷品」一樣。
\item 可以方便的排版數學公式。
\item
用戶只需要學一些聲明文檔邏輯結構的簡單易懂的命令,而不必對文檔的實際版面修修補補。
\item 可以容易的生成像腳註、引用、目錄和參考文獻等很多複雜的結構。
\item 很多不被基本 \LaTeX 支持的排版工作,可以由添加免費的宏包來完成。例如,支持在文件中插入 \PSi{} 格式圖像的宏包及排版
符合各類準確標準的參考文獻的宏包等。很多這類宏包在 \companion 中都有說明。
\item
\LaTeX{} 鼓勵作者按照合理的結構寫作,因為 \LaTeX{} 就是通過指明文檔結構來進行排版工作的。
\item \TeX{},作為 \LaTeXe 的排版引擎,不僅免費,而且具有很高的可移植性,幾乎可以在任何硬件平台上運行。

%
% Add examples ...
%
\end{itemize}

\medskip

%\noindent\LaTeX{} also has some disadvantages, and I guess it's a
%bit difficult for me to find any sensible ones, though I am sure
%other people can tell you hundreds \texttt{;-)}

\noindent\LaTeX{} 也有一些不足之處。儘管我可以確定別人可以列出幾百條,我自己卻很難找到一些比較理智的 \texttt{;-)}

%\begin{itemize}
%\item \LaTeX{} does not work well for people who have sold their
%  souls \ldots
%\item Although some parameters can be adjusted within a predefined
%  document layout, the design of a whole new layout is difficult and
%  takes a lot of time.\footnote{Rumour says that this is one of the
%    key elements that will be addressed in the upcoming \LaTeX 3
%    system.}\index{LaTeX3@\LaTeX 3}
%\item It is very hard to write unstructured and disorganized documents.
%\item Your hamster might, despite some encouraging first steps, never be
%able to fully grasp the concept of Logical Markup.
%\end{itemize}

\begin{itemize}
\item 沒有原則的人不能使用 \LaTeX{} 很好地工作……
\item 儘管可以調節預先定義好的文檔版面佈局中的一些參數,但設計一個全新的版面還是很睏難的,並會耗費大量時
間\footnote{傳聞這將是未來的 \LaTeX 3 系統中的一個重要組成部分。}。
\index{LaTeX3@\LaTeX 3}
\item 很難用 \LaTeX{} 來寫結構不明、組織無序的文檔。
\item 即使有一個令人鼓舞的開端,你也可能無法完全掌握其精髓。
\end{itemize}

%\section{\LaTeX{} Input Files}
\section{\LaTeX{} 源文件}

%The input for \LaTeX{} is a plain \texttt{ASCII} text file. You can create it
%with any text editor. It contains the text of the document, as well as
%the commands that tell \LaTeX{} how to typeset the text.

\LaTeX{} 源文件為普通的 \texttt{ASCII} 文件,你可以使用任何文本編輯器來創建。\LaTeX{} 源文件不僅包含了
要排版的文本,而且也包含了告訴 \LaTeX{} 如何排版這些文本內容的命令。

%\subsection{Spaces}
\subsection{空白距離}

%``Whitespace'' characters, such as blank or tab, are
%treated uniformly as ``\wi{space}'' by \LaTeX{}. \emph{Several
%  consecutive} \wi{whitespace} characters are treated as \emph{one}
%``space.''  Whitespace at the start of a line is generally ignored, and
%a single line break is treated as ``whitespace.''
%\index{whitespace!at the start of a line}

空格和製表符等空白字符在 \LaTeX{} 中被看作相同的空白距離 (\wi{space})。{\textbf
多}個連續的空白字符等同於\textbf{一}個空白字符。在句首的
空白距離一般會被忽略,單個空行也被認為是一個「空白距離」。
\index{whitespace!at the start of a line}


%An empty line between two lines of text defines the end of a
%paragraph. \emph{Several} empty lines are treated the same as
%\emph{one} empty line. The text below is an example. On the left hand
%side is the text from the input file, and on the right hand side is the
%formatted output.

兩行文本間的空白行標誌著上段的結束和下段的開始。{\textbf
多}個空白行的作用等同於\textbf{一}個空白行。下面便是一個例子,左邊
是源文件中的文本,右邊是排版後的結果。

\begin{example}
It does not matter whether you
enter one or several     spaces
after a word.

An empty line starts a new
paragraph.
\end{example}

%\subsection{Special Characters}
\subsection{特殊字符}

%The following symbols are \wi{reserved characters} that either have
%a special meaning under \LaTeX{} or are not available in all the
%fonts. If you enter them directly in your text, they will normally
%not print, but rather coerce \LaTeX{} to do things you did not
%intend.

下面的這些字符是 \LaTeX{} 中的保留字符 (\wi{reserved
characters}),它們或在 \LaTeX{} 中有特殊的意義,或不一定存在於所有字庫中。如果你直接在文本中
輸入這些字符,通常它們不會被輸出,而且還會導致 \LaTeX{} 做一些你不希望發生的事情。
\begin{code}
\verb.#  $  %  ^  &  _  {  }     \ . %$
\end{code}

%As you will see, these characters can be used in your documents all
%the same by adding a prefix backslash:

如你看到的,在這些字符前加上反斜線,它們就可以正常的輸出到文檔中。

\begin{example}
\# \$ \% \^{} \& \_ \{ \} \ {}
\end{example}

%The other symbols and many more can be printed with special commands
%in mathematical formulae or as accents. The backslash character
%$\backslash$ can \emph{not} be entered by adding another backslash
%in front of it (\verb|\\|); this sequence is used for
%line breaking.\footnote{Try the \texttt{\$}\ci{backslash}\texttt{\$} command instead. It
%  produces a `$\backslash$'.}

其他一些特殊符號可以由數學環境中的特殊命令或重音命令得到。反斜線 $\backslash$ {\textbf
不}能通過在其前面加另一個反斜線得
到 (\verb|\\|);這是一個用來換行的命令\footnote{試試 \texttt{\$}\ci{backslash}\texttt{\$} 命令,它將生成一個 `$\backslash$'。}。

%\subsection{\LaTeX{} Commands}
\subsection{\LaTeX{} 命令}

%\LaTeX{} \wi{commands} are case sensitive, and take one of the
%following two formats:

\LaTeX{} 命令 (\wi{commands}) 是大小寫敏感的,有以下兩種格式:
%\begin{itemize}
%\item They start with a \wi{backslash} \verb|\| and then have a name
% consisting of letters only. Command names are terminated by a
% space, a number or any other `non-letter.'
%\item They consist of a backslash and exactly one non-letter.
%\end{itemize}
\begin{itemize}
\item 以一個反斜線 (\wi{backslash}) \verb|\| 開始,命令名只由字母組成。命令名後的空格符、數字或任何非字母的字符都標誌著該命令的結束。
\item 由一個反斜線和非字母的字符組成。
\end{itemize}

%
% \\* doesn't comply !
%

%
% Can \3 be a valid command ? (jacoboni)
%
\label{whitespace}

%\LaTeX{} ignores whitespace after commands. If you want to get a
%\index{whitespace!after commands}space after a command, you have to
%put either \verb|{}| and a blank or a special spacing command after the
%command name. The \verb|{}| stops \LaTeX{} from eating up all the space after
%the command name.

\LaTeX{} 忽略命令之後的空白字符。如果你希望在命令後得到一個空
格,可以在命令後加上 \verb|{}| 和一個空格,或加上一個特殊的空格命令。\verb|{}| 將阻止 \LaTeX{} 吃掉命令後的所有空格。
\index{whitespace!after commands}

\begin{example}
I read that Knuth divides the
people working with \TeX{} into
\TeX{}nicians and \TeX perts.\\
Today is \today.
\end{example}

%Some commands need a \wi{parameter}, which has to be given between
%\wi{curly braces} \verb|{ }| after the command name. Some commands support
%\wi{optional parameters}, which are added after the command name in
%\wi{square brackets} \verb|[ ]|. The next examples use some \LaTeX{}
%commands. Don't worry about them; they will be explained later.

有些命令需要一個參數 (\wi{parameter}),該參數用花括號 (\wi{curly
braces}) \verb|{ }| 括住並寫在命令的後面。一些命令支持可選參數 (\wi{optional
parameters}),可 選參數可用方括號 (\wi{square brackets}) \verb|[ ]| 括住,
然後寫在命令的後面。下面的例子中使用了一些 \LaTeX{} 命令,不要著急,後面
將解釋它們的含義。

\begin{example}
You can \textsl{lean} on me!
\end{example}
\begin{example}
Please, start a new line
right here!\newline
Thank you!
\end{example}

%\subsection{Comments}
\subsection{註釋}
\index{comments}

%When \LaTeX{} encounters a \verb|%| character while processing an input file,
%it ignores the rest of the present line, the line break, and all
%whitespace at the beginning of the next line.

當 \LaTeX{} 處理一個源文件時,如果遇到一個百分號 \verb|%|,\LaTeX{} 將忽略 \verb|%| 後的該行內容,換行符以及下
一行前的空白字符。

%This can be used to write notes into the input file, which will not show up
%in the printed version.

我們可以據此在源文件中寫一些註釋,而且這些註釋並不會出現在最後的排版結果中。

\begin{example}
This is an % stupid
% Better: instructive <----
example: Supercal%
              ifragilist%
    icexpialidocious
\end{example}

%The \texttt{\%} character can also be used to split long input lines where no
%whitespace or line breaks are allowed.

符號 \texttt{\%} 也可以用來斷開不能含有空白字符或換行符的較長輸入內容。

%For longer comments you could use the \ei{comment} environment
%provided by the \pai{verbatim} package. This means, that you have to add the
%line \verb|\usepackage{verbatim}| to the preamble of your document as
%explained below before you can use this command.

如果註釋的內容較長,你可以使用 \pai{verbatim} 宏包提供的 \ei{comment} 環境。當然,在使用該環境前,你要在文檔的
導言區 (後面將會解釋其含義) 加上命令 \verb|\usepackage{verbatim}|。
\begin{example}
This is another
\begin{comment}
rather stupid,
but helpful
\end{comment}
example for embedding
comments in your document.
\end{example}

%Note that this won't work inside complex environments, like math for example.
需要注意的是以上做法在數學環境等複雜環境中不起作用。

%\section{Input File Structure}
\section{源文件的結構}\label{inputfilestructure}

%When \LaTeXe{} processes an input file, it expects it to follow a
%certain \wi{structure}. Thus every input file must start with the
%command

當 \LaTeXe{} 處理源文件時,它希望源文件遵從一定的結構 (\wi{structure})。因此,每個源文件都要以如下命令開始
\begin{code}
\verb|\documentclass{...}|
\end{code}
%This specifies what sort of document you intend to write. After
%that, you can include commands that influence the style of the whole
%document, or you can load \wi{package}s that add new features to the
%\LaTeX{} system. To load such a package you use the command
這條命令指明了你所寫的源文檔的類型。然後,你就可以加入控制整篇文檔樣式的命令,或者載入一些為 \LaTeX{} 增加新特性
的宏包 (\wi{package})。可以用如下命令載入一個宏包
\begin{code}
\verb|\usepackage{...}|
\end{code}

%When all the setup work is done,\footnote{The area between \texttt{\bs
%    documentclass} and \texttt{\bs
%    begin$\mathtt{\{}$document$\mathtt{\}}$} is called the
%  \emph{\wi{preamble}}.} you start the body of the text with the
%command

當完成所有的設置工作後\footnote{在 \texttt{\bs
    documentclass} 和 \texttt{\bs
    begin$\mathtt{\{}$document$\mathtt{\}}$}之間的部分稱作\textbf{導言區} (\wi{preamble})。},你可以用下面的命令開始文檔的主體
\begin{code}
\verb|\begin{document}|
\end{code}

%Now you enter the text mixed with some useful \LaTeX{} commands.  At
%the end of the document you add the

現在你就可以輸入帶有 \LaTeX{} 命令的正文了。在文章末尾使用命令
\begin{code}
\verb|\end{document}|
\end{code}
%command, which tells \LaTeX{} to call it a day. Anything that
%follows this command will be ignored by \LaTeX.
來告訴 \LaTeX{} 文檔已經結束。\LaTeX{} 會忽略此命令後的所有內容。

%Figure \ref{mini} shows the contents of a minimal \LaTeXe{} file. A
%slightly more complicated \wi{input file} is given in
%Figure \ref{document}.

圖 \ref{mini} 顯示的是一個簡單的 \LaTeXe 文檔的結構。一個較為複雜的源文件 (\wi{input
file}) 結構如圖 \ref{document} 所示。

\begin{figure}[!bp]
\begin{lined}{6cm}
\begin{verbatim}
\documentclass{article}
\begin{document}
Small is beautiful.
\end{document}
\end{verbatim}
\end{lined}
\caption{一個簡單的 \LaTeX{} 源文件。} \label{mini}
\end{figure}

\begin{figure}[!bp]
\begin{lined}{10cm}
\begin{verbatim}
\documentclass[a4paper,11pt]{article}
% define the title
\author{H. Partl}
\title{Minimalism}
\begin{document}
% generates the title
\maketitle
% insert the table of contents
\tableofcontents
\section{Some Interesting Words}
Well, and here begins my lovely article.
\section{Good Bye World}
\ldots{} and here it ends.
\end{document}
\end{verbatim}
\end{lined}
\caption[article 類例子。]{article 類 \LaTeX{} 源文件例子,該例中的所有命令後面都會講到。}
\label{document}

\end{figure}

%\section{A Typical Command Line Session}
\section{一個典型的命令行過程}

%I bet you must be dying to try out the neat small \LaTeX{} input file
%shown on page \pageref{mini}. Here is some help:
%\LaTeX{} itself comes without a GUI or
%fancy buttons to press. It is just a program that crunches away at your
%input file. Some \LaTeX{} installations feature a graphical front-end where
%you can click \LaTeX{} into compiling your input file. On other systems
%there might be some typing involved, so here is how to coax \LaTeX{} into
%compiling your input file on a text based system. Please note: this
%description assumes that a working \LaTeX{} installation already sits on
%your computer.\footnote{This is the case with most well groomed Unix
%Systems, and \ldots{} Real Men use Unix, so \ldots{} \texttt{;-)}}

我敢打賭你現在一定非常渴望嘗試第 \pageref{mini} 頁上短小簡潔的 \LaTeX{} 源文件。下面便是一些幫助:\LaTeX{} 本身沒有圖形
用戶界面或漂亮的按鈕,它僅僅是一個處理你提供的源文件的程序。有些 \LaTeX{} 安裝版本提供了一個前端圖形界面,你可以通過點
擊按鈕來編譯你的源文件。其他的一些系統上可能就要使用命令來編譯源文件,下面演示的就是如何在一個基於文本的系統上
讓 \LaTeX{} 編譯你的源文件。需要注意:以下演示的前提是 \LaTeX{} 已經正確的安裝到了你的電腦中\footnote{這是在大部分 Unix 系
統下的情況……高手使用 Unix,所以…… \texttt{;-)}}。

%\begin{enumerate}
%\item
%
%  Edit/Create your \LaTeX{} input file. This file must be plain ASCII
%  text.  On Unix all the editors will create just that. On Windows you
%  might want to make sure that you save the file in ASCII or
%  \emph{Plain Text} format.  When picking a name for your file, make
%  sure it bears the extension \eei{.tex}.
%
%\item
%
%Run \LaTeX{} on your input file. If successful you will end up with a
%\texttt{.dvi} file. It may be necessary to run \LaTeX{} several times to get
%the table of contents and all internal references right. When your input
%file has a bug \LaTeX{} will tell you about it and stop processing your
%input file. Type \texttt{ctrl-D} to get back to the command line.
%\begin{lscommand}
%\verb+latex foo.tex+
%\end{lscommand}
%
%\item
%Now you may view the DVI file. There are several ways to do that. You can show the file on screen with
%\begin{lscommand}
%\verb+xdvi foo.dvi &+
%\end{lscommand}
%This only works on Unix with X11. If you are on Windows you might want to try \texttt{yap} (yet another previewer).
%
%You can also convert the dvi file to \PSi{} for printing or viewing with Ghostscript.
%\begin{lscommand}
%\verb+dvips -Pcmz foo.dvi -o foo.ps+
%\end{lscommand}
%
%If you are lucky your \LaTeX{} system even comes with the \texttt{dvipdf} tool, which allows
%you to convert your \texttt{.dvi} files straight into pdf.
%\begin{lscommand}
%\verb+dvipdf foo.dvi+
%\end{lscommand}
%
%\end{enumerate}

\begin{enumerate}
\item
創建並編輯你的源文件。源文件必須是普通的 ASCII 格式。在 Unix 系統下,所有的編輯器都可以創建這樣的文件。在 Windows 系統下,你必須確保文件以 ASCII 或\textbf{普通文本}格式保存。當選取你源文件的文件名時,確保它的擴展名是 \eei{.tex}。
\item
運行 \LaTeX{} 編譯你的源文件。如果成功的話,你將會得到一個 \texttt{.dvi} 文件。為了得到目錄和所有的內部引用,可能要多次運行 \LaTeX{}。當源文件中存在錯誤時,\LaTeX{} 會告訴你錯誤並停止處理源文件。輸入 \texttt{ctrl-D} 可以返回到命令行。
\begin{lscommand}
\verb+latex foo.tex+
\end{lscommand}
\item
現在可以通過幾種方法來預覽得到的 DVI 文件。你可以使用下列命令將文件顯示到螢幕上
\begin{lscommand}
\verb+xdvi foo.dvi &+
\end{lscommand}
這種方法只適用於安裝了 X11 的 Unix 系統。如果你使用的是 Windows 系統,可以使用 \texttt{yap} 來預覽(或其他預覽程序)。

你也可以使用 Ghostscript 將 dvi 文件轉換成 \PSi{} 文件來列印或預覽。
\begin{lscommand}
\verb+dvips -Pcmz foo.dvi -o foo.ps+
\end{lscommand}
如果你的 \LaTeX{} 系統中帶有 \texttt{dvipdf} 工具的話,就可以直接將 \texttt{.dvi} 文件轉換成 pdf 文件。
\begin{lscommand}
\verb+dvipdf foo.dvi+
\end{lscommand}
\end{enumerate}

%\section{The Layout of the Document}
\section{文檔佈局}

%\subsection {Document Classes}\label{sec:documentclass}
\subsection{文檔類}\label{sec:documentclass}

%The first information \LaTeX{} needs to know when processing an
%input file is the type of document the author wants to create. This
%is specified with the \ci{documentclass} command.

當 \LaTeX{} 處理源文件時,首先需要知道的就是作者所要創建的文檔類型。文檔類型可由 \ci{documentclass} 命令來指定。
\begin{lscommand}
\ci{documentclass}\verb|[|\emph{options}\verb|]{|\emph{class}\verb|}|
\end{lscommand}
%\noindent Here \emph{class} specifies the type of document to be
%created. Table \ref{documentclasses} lists the document classes
%explained in this introduction. The \LaTeXe{} distribution provides
%additional classes for other documents, including letters and
%slides.  The \emph{\wi{option}s} parameter customises the behaviour
%of the document class. The options have to be separated by commas.
%The most common options for the standard document classes are listed
%in Table \ref{options}.

\noindent
\emph{class} 指定想要的文檔類型。表 \ref{documentclasses} 給出了一些文檔類型的解釋。\LaTeXe{} 發行版中還提供了其他一些文檔類,像信件和幻燈片等。通過 \emph{\wi{option}s} 參數可以定製文檔類的屬性。不同的選項之間須用逗號
隔開。標準文檔類的最常用選項如表 \ref{options} 所示。

%\begin{table}[!bp]
%\caption{Document Classes.} \label{documentclasses}
%\begin{lined}{\textwidth}
%\begin{description}
%
%\item [\normalfont\texttt{article}] for articles in scientific journals, presentations,
%  short reports, program documentation, invitations, \ldots
%  \index{article class}
%\item [\normalfont\texttt{proc}] a class for proceedings based on the article class.
%  \index{proc class}
%\item [\normalfont\texttt{minimal}] is as small as it can get.
%It only sets a page size and a base font. It is mainly used for debugging
%purposes.
%  \index{minimal class}
%\item [\normalfont\texttt{report}] for longer reports containing several chapters, small
%  books, PhD theses, \ldots \index{report class}
%\item [\normalfont\texttt{book}] for real books \index{book class}
%\item [\normalfont\texttt{slides}] for slides. The class uses big sans serif
%  letters. You might want to consider using Foil\TeX{}\footnote{%
%        \CTANref|macros/latex/contrib/supported/foiltex|} instead.
%        \index{slides class}\index{foiltex}
%\end{description}
%\end{lined}
%\end{table}
%
%\begin{table}[!bp]
%\caption{Document Class Options.} \label{options}
%\begin{lined}{\textwidth}
%\begin{flushleft}
%\begin{description}
%\item[\normalfont\texttt{10pt}, \texttt{11pt}, \texttt{12pt}] \quad Sets the size
%  of the main font in the document. If no option is specified,
%  \texttt{10pt} is assumed.  \index{document font size}\index{base
%    font size}
%\item[\normalfont\texttt{a4paper}, \texttt{letterpaper}, \ldots] \quad Defines
%  the paper size. The default size is \texttt{letterpaper}. Besides
%  that, \texttt{a5paper}, \texttt{b5paper}, \texttt{executivepaper},
%  and \texttt{legalpaper} can be specified.  \index{legal paper}
%  \index{paper size}\index{A4 paper}\index{letter paper} \index{A5
%    paper}\index{B5 paper}\index{executive paper}
%
%\item[\normalfont\texttt{fleqn}] \quad Typesets displayed formulae left-aligned
%  instead of centred.
%
%\item[\normalfont\texttt{leqno}] \quad Places the numbering of formulae on the
%  left hand side instead of the right.
%
%\item[\normalfont\texttt{titlepage}, \texttt{notitlepage}] \quad Specifies
%  whether a new page should be started after the \wi{document title}
%  or not. The \texttt{article} class does not start a new page by
%  default, while \texttt{report} and \texttt{book} do.  \index{title}
%
%\item[\normalfont\texttt{onecolumn}, \texttt{twocolumn}] \quad Instructs \LaTeX{} to typeset the
%  document in \wi{one column} or \wi{two column}s.
%
%\item[\normalfont\texttt{twoside, oneside}] \quad Specifies whether double or
%  single sided output should be generated. The classes
%  \texttt{article} and \texttt{report} are \wi{single sided} and the
%  \texttt{book} class is \wi{double sided} by default. Note that this
%  option concerns the style of the document only. The option
%  \texttt{twoside} does \emph{not} tell the printer you use that it
%  should actually make a two-sided printout.
%\item[\normalfont\texttt{landscape}] \quad Changes the layout of the document to print in landscape mode.
%\item[\normalfont\texttt{openright, openany}] \quad Makes chapters begin either
%  only on right hand pages or on the next page available. This does
%  not work with the \texttt{article} class, as it does not know about
%  chapters. The \texttt{report} class by default starts chapters on
%  the next page available and the \texttt{book} class starts them on
%  right hand pages.
%
%\end{description}
%\end{flushleft}
%\end{lined}
%\end{table}

\begin{table}[!bp]
\caption{文檔類。} \label{documentclasses}
\begin{lined}{\textwidth}
\begin{description}

\item [\normalfont\texttt{article}] 排版科學期刊、演示文檔、短報告、程序文檔、邀請函……
  \index{article class}
\item [\normalfont\texttt{proc}] 一個基於 article 的會議文集類。
  \index{proc class}
\item [\normalfont\texttt{minimal}] 非常小的文檔類。只設置了頁面尺寸和基本字體。主要用來查錯。
  \index{minimal class}
\item [\normalfont\texttt{report}] 排版多章節長報告、短篇書籍、博士論文……\index{report class}
\item [\normalfont\texttt{book}] 排版書籍。\index{book class}
\item [\normalfont\texttt{slides}] 排版幻燈片。該文檔類使用大號 sans
serif 字體。也可以選用 Foil\TeX{}\footnote{%
        \CTANref|macros/latex/contrib/supported/foiltex|} 來得到相同的效果。
        \index{slides class}\index{foiltex}
\end{description}
\end{lined}
\end{table}

\begin{table}[!bp]
\caption{文檔類選項。} \label{options}
\begin{lined}{\textwidth}
\begin{flushleft}
\begin{description}
\item[\normalfont\texttt{10pt}, \texttt{11pt}, \texttt{12pt}] \quad 設置文檔中所使用的字體的大小。如果該項沒有指
  定,默認使用 \texttt{10pt} 字體。\index{document font size}\index{base
    font size}
\item[\normalfont\texttt{a4paper}, \texttt{letterpaper}, \ldots] \quad 定義紙張的尺寸。缺省設置為 \texttt{letterpaper}。此
  外,還可以使用 \texttt{a5paper}, \texttt{b5paper}, \texttt{executivepaper} 以及 \texttt{legalpaper}。\index{legal paper}
  \index{paper size}\index{A4 paper}\index{letter paper} \index{A5
    paper}\index{B5 paper}\index{executive paper}

\item[\normalfont\texttt{fleqn}] \quad 設置行間公式為左對齊,而不是居中對齊。

\item[\normalfont\texttt{leqno}] \quad
設置行間公式的編號為左對齊,而不是右對齊。

\item[\normalfont\texttt{titlepage}, \texttt{notitlepage}] \quad 指定是否在文檔標題 (\wi{document title}) 後另起一
  頁。\texttt{article} 文檔類缺省設置為不開始新頁,\texttt{report} 和 \texttt{book} 類則相反。\index{title}

\item[\normalfont\texttt{onecolumn}, \texttt{twocolumn}] \quad
  \LaTeX{} 以單欄 (\wi{one column}) 或雙欄 (\wi{two
  column}) 的方式來排版文檔。

\item[\normalfont\texttt{twoside}, \texttt{oneside}] \quad 指定文檔為雙面或單面列印格式。\texttt{article} 和 \texttt{report} 類為
  單面 (\wi{single sided}) 格式,\texttt{book} 類缺省為雙面 (\wi{double sided}) 格式。注意該選項只是作用於文檔樣式,而\textbf{不會}通知列印機以雙面格式列印文檔。

\item[\normalfont\texttt{landscape}] \quad
將文檔的列印輸出佈局設置為 landscape 模式。

\item[\normalfont\texttt{openright}, \texttt{openany}] \quad 決定新的一章僅在奇數頁開始還是在下一頁開始。在文檔類型為 \texttt{article} 時該選項不起作用,因為該類中沒有定義「章」 (chapter)。 \texttt{report} 類默認在下一頁開始新一章而 \texttt{book} 類的新一章總是在奇數頁開始。

\end{description}
\end{flushleft}
\end{lined}
\end{table}

%Example: An input file for a \LaTeX{} document could start with the
%line

例子:一個 \LaTeX{} 源文件以下面一行開始
\begin{code}
\ci{documentclass}\verb|[11pt,twoside,a4paper]{article}|
\end{code}
%which instructs \LaTeX{} to typeset the document as an \emph{article}
%with a base font size of \emph{eleven points}, and to produce a
%layout suitable for \emph{double sided} printing on \emph{A4 paper}.
這條命令會引導 \LaTeX{} 使用 \emph{article} 格式、\textbf{11 磅大小的字體}來排版該文檔,並得到在 \emph{A4} 紙上\textbf{雙面列印}的效果。 \pagebreak[2]

%\subsection{Packages}
\subsection{宏包}
\index{package} %While writing your document, you will probably find
%that there are some areas where basic \LaTeX{} cannot solve your
%problem. If you want to include \wi{graphics}, \wi{coloured text} or
%source code from a file into your document, you need to enhance the
%capabilities of \LaTeX.  Such enhancements are called packages.
%Packages are activated with the

排版文檔時,你可能會發現某些時候基本的 \LaTeX{} 並不能解決你的問題。如果想插入圖形 (\wi{graphics})、
彩色文本 (\wi{coloured text}) 或源代碼到你的文檔中,你就
需要使用宏包來增強 \LaTeX{} 的功能。可使用如下命令調用宏包
\begin{lscommand}
\ci{usepackage}\verb|[|\emph{options}\verb|]{|\emph{package}\verb|}|
\end{lscommand}
\noindent%command, where \emph{package} is the name of the package and
%\emph{options} is a list of keywords that trigger special features in
%the package. Some packages come with the \LaTeXe{} base distribution
%(See Table \ref{packages}). Others are provided separately. You may
%find more information on the packages installed at your site in your
%\guide. The prime source for information about \LaTeX{} packages is \companion.
%It contains descriptions on hundreds of packages, along with
%information of how to write your own extensions to \LaTeXe.
這裡 \emph{package} 是宏包的名稱,\emph{options} 是用來啟動宏包特殊功能的一組關鍵詞。很多宏包隨 \LaTeX{} 基本發行版一起
發佈 (見表 \ref{packages}),其他的則單獨發佈。你可以在所安裝的 \LaTeX{} 系統中找到更多的宏包相關信息。\companion 提供了關
於宏包的重要信息,它包含了數百個宏包的描述及如何寫作自己的 \LaTeXe{} 擴展的信息。

%Modern \TeX{} distributions come with a large number of packages
%preinstalled. If you are working on a Unix system, use the command
%\texttt{texdoc} for accessing package documentation.

現代的 \TeX 發行版包含了大量免費的宏包。如果你使用的是 Unix 系統,可以使用命令 \texttt{texdoc} 搜索宏包的說明文檔。
%\begin{table}[btp]
%\caption{Some of the Packages Distributed with \LaTeX.} \label{packages}
%\begin{lined}{\textwidth}
%\begin{description}
%\item[\normalfont\pai{doc}] Allows the documentation of \LaTeX{} programs.\\
% Described in \texttt{doc.dtx}\footnote{This file should be installed
%   on your system, and you should be able to get a \texttt{dvi} file
%   by typing \texttt{latex doc.dtx} in any directory where you have
%   write permission. The same is true for all the
%   other files mentioned in this table.}  and in \companion.
%
%\item[\normalfont\pai{exscale}] Provides scaled versions of the
%  math extension  font.\\
%  Described in \texttt{ltexscale.dtx}.
%
%\item[\normalfont\pai{fontenc}] Specifies which \wi{font encoding}
%  \LaTeX{} should use.\\
%  Described in \texttt{ltoutenc.dtx}.
%
%\item[\normalfont\pai{ifthen}] Provides commands of the form\\
%  `if\ldots then do\ldots otherwise do\ldots.'\\ Described in
%  \texttt{ifthen.dtx} and \companion.
%
%\item[\normalfont\pai{latexsym}] To access the \LaTeX{} symbol
%  font, you should use the \texttt{latexsym} package. Described in
%  \texttt{latexsym.dtx} and in \companion.
%
%\item[\normalfont\pai{makeidx}] Provides commands for producing
%  indexes.  Described in section \ref{sec:indexing} and in \companion.
%
%\item[\normalfont\pai{syntonly}] Processes a document without
%  typesetting it.
%
%\item[\normalfont\pai{inputenc}] Allows the specification of an
%  input encoding such as ASCII, ISO Latin-1, ISO Latin-2, 437/850 IBM
%  code pages,  Apple Macintosh, Next, ANSI-Windows or user-defined one.
%  Described in \texttt{inputenc.dtx}.
%\end{description}
%\end{lined}
%\end{table}

\begin{table}[btp]
\caption{隨 \LaTeX 一起發行的宏包。} \label{packages}
\begin{lined}{\textwidth}
\begin{description}
\item[\normalfont\pai{doc}] 排版 \LaTeX{} 的說明文檔。具體描述見 \texttt{doc.dtx}\footnote{你的系統中應該安裝了該文
   件,輸入命令 \texttt{latex doc.dtx} 處理該文件可得到一個 \texttt{dvi} 文件。類似的方法適用於本表格中的其
   他 \texttt{.dtx} 文件。} 及 \companion。

\item[\normalfont\pai{exscale}] 提供了按比例伸縮的數學擴展字體。\\
  具體描述見 \texttt{ltexscale.dtx}。

\item[\normalfont\pai{fontenc}] 指明使用哪種 \LaTeX{} 字體編碼 (\wi{font
  encoding})。\\
  具體描述見 \texttt{ltoutenc.dtx}。

\item[\normalfont\pai{ifthen}] 提供如下形式的命令\\
  `if \ldots then do \ldots otherwise do \ldots.'\\具體描述見 
  \texttt{ifthen.dtx} 及 \companion。

\item[\normalfont\pai{latexsym}]
  提供 \LaTeX{} 符號字體。具體描述見 \texttt{latexsym.dtx} 及 \companion。

\item[\normalfont\pai{makeidx}]
  提供排版索引的命令。具體描述見第 \ref{sec:indexing} 節及 \companion。

\item[\normalfont\pai{syntonly}] 編譯文檔而不生成 dvi 文件 (常用於查錯)。

\item[\normalfont\pai{inputenc}] 指明使用哪種輸入編碼,如 ASCII, ISO Latin-1, ISO Latin-2, 437/850 IBM
  code pages,  Apple Macintosh, Next,
  ANSI-Windows 或用戶自定義編碼。
  具體描述見 \texttt{inputenc.dtx}。
\end{description}
\end{lined}
\end{table}

%\subsection{Page Styles}
\subsection{頁面樣式}

%\LaTeX{} supports three predefined \wi{header}/\wi{footer}
%combinations---so-called \wi{page style}s. The \emph{style}
%parameter of the
\index{page
style!plain@\texttt{plain}}\index{plain@\texttt{plain}} \index{page
style!headings@\texttt{headings}}\index{headings@texttt{headings}}
\index{page style!empty@\texttt{empty}}\index{empty@\texttt{empty}}

\LaTeX{} 支持三種預定義的頁眉/頁腳 (\wi{header}/\wi{footer}) 樣式,稱為頁面樣式 (\wi{page
style})。如下命令
\begin{lscommand}
\ci{pagestyle}\verb|{|\emph{style}\verb|}|
\end{lscommand}
\noindent%command defines which one to use.
%Table \ref{pagestyle}
%lists the predefined page styles.
中的 \emph{style} 參數確定了使用哪一種頁面樣式。表 \ref{pagestyle} 列出了預定義的頁面樣式。

%\begin{table}[!hbp]
%\caption{The Predefined Page Styles of \LaTeX.} \label{pagestyle}
%\begin{lined}{\textwidth}
%\begin{description}
%
%\item[\normalfont\texttt{plain}] prints the page numbers on the bottom
%  of the page, in the middle of the footer. This is the default page
%  style.
%
%\item[\normalfont\texttt{headings}] prints the current chapter heading
%  and the page number in the header on each page, while the footer
%  remains empty.  (This is the style used in this document)
%\item[\normalfont\texttt{empty}] sets both the header and the footer
%  to be empty.
%
%\end{description}
%\end{lined}
%\end{table}
\begin{table}[!hbp]
\caption{\LaTeX 預定義的頁面樣式。} \label{pagestyle}
\begin{lined}{\textwidth}
\begin{description}

\item[\normalfont\texttt{plain}] 在頁腳正中顯示頁碼。這是頁面樣式的缺省設置。

\item[\normalfont\texttt{headings}]
在頁眉中顯示章節名及頁碼,頁腳空白。(本文即採用此樣式)
\item[\normalfont\texttt{empty}] 將頁眉頁腳都設為空白。

\end{description}
\end{lined}
\end{table}

%It is possible to change the page style of the current page
%with the command
\hyphenation{Companion}
可以通過如下命令來改變當前頁面的頁面樣式
\begin{lscommand}
\ci{thispagestyle}\verb|{|\emph{style}\verb|}|
\end{lscommand}
%A description how to create your own
%headers and footers can be found in \companion{} and in section \ref{sec:fancy} on page \pageref{sec:fancy}.
如何創建自定義頁眉頁腳的說明可以參見 \companion{} 及第 \pageref{sec:fancy} 頁的第 \ref{sec:fancy} 節。
%
% Pointer to the Fancy headings Package description !
%

%\section{Files You Might Encounter}
\section{各類 \LaTeX{} 文件}

%When you work with \LaTeX{} you will soon find yourself in a maze of
%files with various \wi{extension}s and probably no clue. The
%following list explains the various \wi{file types} you might
%encounter when working with \TeX{}. Please note that this table does
%not claim to be a complete list of extensions, but if you find one
%missing that you think is important, please drop me a line.
使用 \LaTeX{} 時,你可能很快發現自己置身於各種不同擴展名 (\wi{extension}) 或毫無線索的文件形成的迷宮之中。下面的列表解釋了
在使用 \LaTeX{} 時可能遇到的文件類型。要注意的是,下表不是所有的擴展名列表,如果你發現有重要的文件類型沒有收錄
進來,請通知我。
%\begin{description}
%
%\item[\eei{.tex}] \LaTeX{} or \TeX{} input file. Can be compiled with
%  \texttt{latex}.
%\item[\eei{.sty}] \LaTeX{} Macro package. This is a file you can load
%  into your \LaTeX{} document using the \ci{usepackage} command.
%\item[\eei{.dtx}] Documented \TeX{}. This is the main distribution
%  format for \LaTeX{} style files. If you process a .dtx file you get
%  documented macro code of the \LaTeX{} package contained in the .dtx
%  file.
%\item[\eei{.ins}] The installer for the files contained in the
%  matching .dtx file. If you download a \LaTeX{} package from the net,
%  you will normally get a .dtx and a .ins file. Run \LaTeX{} on the
%  .ins file to unpack the .dtx file.
%\item[\eei{.cls}] Class files define what your document looks
%  like. They are selected with the \ci{documentclass} command.
%\item[\eei{.fd}] Font description file telling  \LaTeX{} about new fonts.
%\end{description}
\begin{description}

\item[\eei{.tex}]
  \LaTeX{} 或 \TeX{} 源文件。可以使用 \texttt{latex} 命令編譯。
\item[\eei{.sty}]
  \LaTeX{} 宏包文件。可以使用 \ci{usepackage} 命令將宏包文件載入到你的 \LaTeX{} 文檔中。
\item[\eei{.dtx}]
  文檔化 \TeX{} 文件。這是 \LaTeX{} 宏包文件的主要發佈格式。如果編譯 \texttt{.dtx} 文檔,將會得到其中包含的 \LaTeX{} 宏包文件
  的文檔化宏代碼。
\item[\eei{.ins}] 對應 \texttt{.dtx} 文件的安裝文件。如果你從網上下載了一個 \LaTeX{} 的宏包文件,其中一般會包含一
  個 \texttt{.dtx} 文件和一個 \texttt{.ins} 文件。使用 \LaTeX{} 處理 \texttt{.ins} 文件可以解開 \texttt{.dtx} 文件。
\item[\eei{.cls}] 定義文檔外觀形式的類文件,可以通過使用 \ci{documentclass} 命令選取。
\item[\eei{.fd}] 字體描述文件,可以告訴 \LaTeX{} 有關新字體的信息。
\end{description}
%The following files are generated when you run \LaTeX{} on your input
%file:
下面這些文件是使用 \LaTeX{} 處理源文件時產生的:

%\begin{description}
%\item[\eei{.dvi}] Device Independent File. This is the main result of a \LaTeX{}
%  compile run. You can look at its content with a DVI previewer
%  program or you can send it to a printer with \texttt{dvips} or a
%  similar application.
%\item[\eei{.log}] Gives a detailed account of what happened during the
%  last compiler run.
%\item[\eei{.toc}] Stores all your section headers. It gets read in for the
%  next compiler run and is used to produce the table of content.
%\item[\eei{.lof}] This is like .toc but for the list of figures.
%\item[\eei{.lot}] And again the same for the list of tables.
%\item[\eei{.aux}] Another file that transports information from one
%  compiler run to the next. Among other things, the .aux file is used
%  to store information associated with cross-references.
%\item[\eei{.idx}] If your document contains an index. \LaTeX{} stores all
%  the words that go into the index in this file. Process this file with
%  \texttt{makeindex}. Refer to section \ref{sec:indexing} on
%  page \pageref{sec:indexing} for more information on indexing.
%\item[\eei{.ind}] The processed .idx file, ready for inclusion into your
%  document on the next compile cycle.
%\item[\eei{.ilg}] Logfile telling what \texttt{makeindex} did.
%\end{description}
\begin{description}
\item[\eei{.dvi}] 設備無關文件。這是運行 \LaTeX{} 編譯的主要結果。你可以使用 DVI 預覽器預覽其內容或使
  用 \texttt{dvips} 或其他程序輸出到列印機。
\item[\eei{.log}] 記錄了上次編譯時的詳細信息。
\item[\eei{.toc}]
  儲存了所有的章節標題。下次編譯時將讀取該文件並生成目錄。
\item[\eei{.lof}] 和 \texttt{.toc} 文件類似,可生成圖形目錄。
\item[\eei{.lot}] 和 \texttt{.toc} 文件類似,可生成表格目錄。
\item[\eei{.aux}] 用來向下次編譯傳遞信息的輔助文件。主要儲存交叉引用的相關信息。
\item[\eei{.idx}] 如果文檔中包含索引,\LaTeX{} 將使用該文件存儲所有的索引詞條。此文件需要使
  用 \texttt{makeindex} 處理,詳見位於 \pageref{sec:indexing} 頁的第 \ref{sec:indexing} 節。
\item[\eei{.ind}] 處理過的 \texttt{.idx} 文件。下次編譯時將讀入到你的文檔中。
\item[\eei{.ilg}]
  和 \texttt{.log} 文件類似,記錄了 \texttt{makeindex} 命令運行的詳細信息。
\end{description}

% Package Info pointer
%
%



%
% Add Info on page-numbering, ...
% \pagenumbering

%\section{Big Projects}
\section{大型項目}
%When working on big documents, you might want to split the input
%file into several parts. \LaTeX{} has two commands that help you to
%do that.

當處理大型文檔時,最好將文檔分割成為幾部分。\LaTeX{} 有兩個命令可以幫助你完成這項工作。

\begin{lscommand}
\ci{include}\verb|{|\emph{filename}\verb|}|
\end{lscommand}
\noindent %You can use this command in the document body to insert the
%contents of another file named \emph{filename.tex}. Note that \LaTeX{}
%will start a new page
%before processing the material input from \emph{filename.tex}.
你可以使用該命令將名為 \emph{filename.tex} 的文檔內容插入到當前文檔中。需要注意的是,在處理插入的 \emph{filename.tex} 文
檔前,\LaTeX{} 會另起一頁。

%The second command can be used in the preamble. It allows you to
%instruct \LaTeX{} to only input some of the \verb|\include|d files.

第二個命令只能在導言區使用。它可以讓 \LaTeX{} 僅讀入某些 \verb|\include| 文件。
\begin{lscommand}
\ci{includeonly}\verb|{|\emph{filename}\verb|,|\emph{filename}%
\verb|,|\ldots\verb|}|
\end{lscommand}
%After this command is executed in the preamble of the document, only
%\ci{include} commands for the filenames that are listed in the
%argument of the \ci{includeonly} command will be executed. Note that
%there must be no spaces between the filenames and the commas.
這條命令在文檔的導言區執行後,在所有的 \ci{include} 命令中,只有文檔名出現在 \ci{includeonly} 的命令參數中的文檔
才會被導入。注意文檔名和逗號之間不能有空格。

%The \ci{include} command starts typesetting the included text on a new
%page. This is helpful when you use \ci{includeonly}, because the
%page breaks will not move, even when some included files are omitted.
%Sometimes this might not be desirable. In this case, you can use the

\ci{include} 命令會在新的一頁上排版載入的文本。當使用 \ci{includeonly} 命令時會很有幫助,因為即使一些載入的文本被忽略,分頁
處也不會發生變化。有些時候可能不希望在新的一頁上排版載入的文本,這時可以使用命令
\begin{lscommand}
\ci{input}\verb|{|\emph{filename}\verb|}|
\end{lscommand}
\noindent% command. It simply includes the file specified.
%No flashy suits, no strings attached.
\ci{input} 命令只是簡單的載入指定的文本,沒有其他限制。

%To make \LaTeX{} quickly check your document you can use the \pai{syntonly}
%package. This makes \LaTeX{} skim through your document only checking for
%proper syntax and usage of the commands, but doesn't produce any (DVI) output.
%As \LaTeX{} runs faster in this mode you may save yourself valuable time.
%Usage is very simple:

如果想讓 \LaTeX{} 快速的檢查文檔中的錯誤,可以使用 \pai{syntonly} 宏包。它可以使 \LaTeX{} 瀏覽整個文檔,檢查語法錯誤
和使用的命令,但並不生成 DVI 輸出。在這種模式下,\LaTeX{} 運行速度很快,可以為你節省寶貴的時間。\pai{syntonly} 宏
包的使用非常簡單:
\begin{verbatim}
\usepackage{syntonly}
\syntaxonly
\end{verbatim}
%When you want to produce pages, just comment out the second line
%(by adding a percent sign).
如果想產生分頁,只要註釋掉第二行即可 (在前面加上一個百分號 \verb|%|)。


%

% Local Variables:
% TeX-master: "lshort2e"
% mode: latex
% mode: flyspell
% End:

